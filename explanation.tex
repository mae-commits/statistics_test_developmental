\documentclass[a4paper,12pt]{ltjarticle}
\usepackage{amsmath, ascmac,amssymb}
\usepackage{enumitem}       % 問題番号のカスタマイズ

\title{現代数理統計学の基礎 確認テスト 発展編 解説}
\date{\empty}

\begin{document}
\maketitle

\begin{enumerate}[label=\textbf{問\arabic*}]
\item 解説
\begin{enumerate}[label=(\arabic*)]
\item 確率関数の条件より、
\begin{align} 
  \int_{-\infty}^{\infty} f_{X}(x) dx &= \int_{-1}^{1} C(1-|x|) dx \notag\\
  &= 2\int_{0}^{1} C(1-x)\hspace{3mm}(\because 偶関数) \notag\\
  &= 2C\left[x-\frac{x^2}{2}\right]_{0}^{1} \notag\\
  &= C = 1
\end{align}
より $C=1$ である。また、分布関数 $F_X(x)$ は $f_X(x)$ の範囲に注意すると、
\begin{itemize}
  \item $x \leq -1$ のとき
    $F_X(x) = 0$
  \item $-1 < x \leq 0$ のとき
  \begin{align}
    F_X(x) 
    &= \int_{-\infty}^{x} f_{X}(t) dt \notag \\
    &= \int_{-1}^{x} (1-|t|) dt \notag\\
    &= \int_{-1}^{x} (1+t) dt \notag\\
    &= \left[t+\frac{t^2}{2}\right]_{-1}^{x} \notag\\
    &= x + \frac{x^2}{2} + \frac{1}{2}
  \end{align}
  \item $0 < x < 1$ のとき
  \begin{align}
    F_X(x) 
    &= \int_{-\infty}^{x} f_{X}(t) dt \notag\\
    &= \int_{-1}^{x} (1-|t|) dt \notag\\
    &= \int_{-1}^{0} (1+t) dt + \int_{0}^{x} (1-t) dt \notag\\
    &= \left[t+\frac{t^2}{2}\right]_{-1}^{0} + \left[t-\frac{t^2}{2}\right]_{0}^{x} \notag\\
    &= \frac{1}{2} + x - \frac{x^2}{2}
  \end{align}
  \item $x \geq 1$ のとき
   $F_X(x) = 1$
\end{itemize}
\item $Y=|X|$ とおくと、$Y$ の分布関数 $F_Y(y)$ は、
\begin{align}
  F_Y(y) 
  &= P(Y \leq y) = P(|X| \leq y) \notag\\
  &= P(-y \leq X \leq y) \notag\\
  &= F_X(y) - F_X(-y) \notag
\end{align}
ここで、$Y=|X|$ であることから、$y \geq 0$ に注意して、(1) から
\begin{itemize}
  \item $0 \leq y < 1$ のとき
  \begin{align}
    F_Y(y) 
    &= F_X(y) - F_X(-y) \notag\\
    &= \int_{-y}^{y} (1 - |t|) dt \notag \\
    &= 2\int_{0}^{y} (1-t) dt \notag \\
    &= 2\left[t-\frac{t^2}{2}\right]_{0}^{y} \notag \\
    &= 2y - y^2
  \end{align}
  \item $y \geq 1$ のとき $F_Y(y) = 1$
\end{itemize}
これらを $y$ で微分することで $Y$ の確率密度関数 $f_Y(y)$ を求めることができ、
\begin{equation}
  f_Y(y) = \left\{
  \begin{aligned}
    &2(1-y)\hspace{3mm} & (0 \leq y < 1) \\
    &0\hspace{3mm} & (otherwise)
  \end{aligned}
  \right.
\end{equation}
\item $Z=X^2$ とおくと、$Z$ の分布関数 $F_Z(z)$ は、
\begin{align}
  F_Z(z) 
  &= P(Z \leq z) \notag\\
  &= P(X^2 \leq z) \notag\\
  &= P(-\sqrt{z} \leq X \leq \sqrt{z}) \notag\\ 
  &= \int_{\sqrt{z}}^{\sqrt{z}} f_X(x) dx \hspace{3mm}(z \geq 0)
\end{align}
であることから、(2) と同様に $z$ の範囲で場合分けすることで
\begin{itemize}
  \item $0 \leq z < 1$ のとき
    \begin{align}
      F_Z{z}
      &= \int_{-\sqrt{z}}^{\sqrt{z}} f_X(x) dx \notag\\
      &= \int_{-\sqrt{z}}^{0} (1+|x|) dx + \int_{0}^{\sqrt{z}} (1-|x|) dx \notag\\
      &= 2\int_{0}^{\sqrt{z}} (1+x) dx \notag
    \end{align}
  \item $z \geq 1$ のとき、$F_Z(z) = 1$
\end{itemize}
これらを $z$ で微分することで $Z$ の確率密度関数 $f_Z(z)$ を求めることができる。
\end{enumerate}
\end{enumerate}
\end{document}